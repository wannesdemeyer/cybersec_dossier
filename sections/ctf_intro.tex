\documentclass[../main.tex]{subfiles}

\begin{document}

\section{Hack The Election CTF}
In the second phase of our project, we're going to make our own Capture The Flag. With all the new technical knowledge we gained during the hacking of the Hack The Box machines, we started to set up our own competition. Techniques, protocols, vulnerabilities and much more have been incorporated into our online platform.

What exactly is a Capture The Flag? Security CTF's have nothing to do with conquering the flag while playing paintball. CTF's are tricky tasks, also called challenges, in which you can learn new hacking techniques. It is a way to learn how to hack and get to know people in the (cyber) security community. The target in most challenges is to find the flag, which always has the following format: "\textit{flag\{ThisIsTheFlag\}}".

Because we didn't want to make a traditional CTF, we first looked for a story on which we could build our challenges. We chose to give the American Elections a second chance. Because of the many conspiracy theories about fraud and manipulation, the American Supreme Court decided to reorganise the election. It's now up to the players to carry out assignments in order to bring their favourite party to power. Each completed challenge brings - depending on the difficulty - a number of points, that brings your favourite candidate closer to the presidency.

Specifically, we have created an online platform that we host on Google Cloud with the help of Docker containers. A player can register and gain access to the map of America. Of course, the aim of this CTF is to win as many states as possible and ultimately be crowned president.

Can't you wait to take a look and help determine the elections? Then take a quick look at \textit{\textbf{https://ctf.verkiezingen.xyz/}} and make sure your favourite candidate wins.

\end{document}
