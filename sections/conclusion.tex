\documentclass[../main.tex]{subfiles}

\begin{document}

\section{Conclusion}

\subsection{Hack The Box}

The last six weeks have been a real rollercoaster of emotions, learning moments and experiences. First of all, the "Hack The Box" part was a special experience with an equally special learning curve. When we started working on these machines, it seemed impossible and unrealistic to crack them. For this part, we've googled a lot to learn techniques, theories and tools. 

We've spent most of our time gaining the foothold. Because of the fact that the footholds were kinda custom made exploits, made it tricky for us to understand what to do. Therefore we've had to develop new skills, gain knowledge in new tools and learn to think in a backwards direction. After gaining the foothold, the process speeded up. Enumeration brought us closer to the user flags and wasn't that difficult. As a team we found it very satisfying to put our theoretical knowhow about enumeration into practical experience. The easiest part was the root flag, it was easily achieved by using a public script or a technology. 

The biggest drawback of the whole HTB experience was the presence of users who didn't follow the rules. It's strictly forbidden to post write ups of the machines, but unfortunately we've found some online guides. Our team was very motivated and we've agreed on not consulting these writeups. In the end, we were a bit disappointed in Hack The Box because of the fact that there were some online guides.

\subsection{Capture The Flag}

As a team, we really underestimated this part of our project. Like the first part, we've had to learn thinking in a reverse direction. It sounds a bit strange, our challenges are simplistic but it was very demanding to devise those ideas. 

First things first, we've composed a theme for our CTF. According to the categories within our platform, we divided the challenges. At this moment, we didn't progress that much as we found it very difficult to come up with good ideas and to merge them with the correct technical methodology. Nevertheless, this process was really educational as we needed to dive deeper into our category.


In the end, we aren't satisfied with the quantity of challenges at the same time we are pleased with the quality of them. We gained fulfilment with the fact that, after these six weeks, we've added quite some technical abilities to our skillset.

\end{document}

